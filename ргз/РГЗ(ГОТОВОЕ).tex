\documentclass[12pt]{article}
%пакеты%
\usepackage{hyperref}
\usepackage[russian]{babel}
\usepackage[utf8]{inputenc}
%пакеты%

%форматирование текста%
\linespread{1.5}
\setlength{\parskip}{12pt}
\setlength{\parindent}{0pt}
%форматирование текста%

\def\l{\linebreak}

\begin{document}

\begin{center}

    \textbf{МИНИСТЕРСТВО ОБРАЗОВАНИЯ И НАУКИ
    РОССИЙСКОЙ ФЕДЕРАЦИИ} \l
    ФЕДЕРАЛЬНОЕ ГОСУДАРСТВЕННОЕ БЮДЖЕТНОЕ ОБЩЕОБРАЗОВАТЕЛЬНОЕ УЧРЕЖДЕНИЕ
    ВЫСШЕГО ОБРАЗОВАНИЯ \l
    \textbf{<<БЕЛГОРОДСКИЙ ГОСУДАРСТВЕННЫЙ
    ТЕХНОЛОГИЧЕСКИЙ УНИВЕРСИТЕТ им В.Г.ШУХОВА>> \l
    (БГТУ им. В.Г.Шухова)}
    \l \l
    Кафедра программного обеспечения вычислительной техники и 
    автоматизированных систем
    \l \l \l  
    Расчётно-графическое задание \l
    дисциплина: Информатика \l
    тема: <<Компьютерные вирусы, их свойства и классификация>> \l
    
\hfill

\begin{flushright}
    Выполнил: ст.группы ПВ-201 \l
    Машуров Дмитрий Русланович \l
    Проверил: Бондаренко Т.В. 
\end{flushright}

    Белгород 2020
    
\end{center}
\thispagestyle{empty}

\tableofcontents
\thispagestyle{empty}

\newpage
\setcounter{page}{1}
\section*{ВВЕДЕНИЕ}
\addcontentsline{toc}{section}{ВВЕДЕНИЕ}

\subsection*{Постановка темы}
\addcontentsline{toc}{subsection}{Постановка задачи}

С приходом 21 века в нашу жизни постепенно проникали различные бытовые устройства, облегчающие жизнь человека: холодильник, микроволновая печь, стиральная машина, телефон... Но самым главным из этих изобретений был компьютер.

Согласно определению, \textbf{компьютер} — устройство или система, способная выполнять заданную, чётко определённую, изменяемую последовательность операций [1].

Появление компьютера в несколько раз облегчило жизнь человека. Он помогает нам в различных областях деятельности - от набора текста до моделирования математических и физических расчётов.

Но, к сожалению, вместе с появлением компьютеров пришли люди, которые начали пользоваться данным изобретением в корыстных целях. Они крадут деньги и конфиденциальную информацию, выводят из строя банковские системы. Делают они это с помощью вредоносных программ, нызываемых \textbf{вирусами}.

\newpage
\subsection*{Актуальность темы}
\addcontentsline{toc}{subsection}{Актуальность темы}

Актуальность данной темы объясняется тем, что с появлением компьютеров и телефонов люди стали хранить практически всю информацию на них. А это в свою очередь очень выгодно мошенниками, которые с помощью компьютерных вирусов проникают на устройства своих жертв и крадут важные данные: адреса, номера телефонов, пароли от аккаунтов, личную информацию.

Но что вообще такое компьютерный вирус? Как он попадает к нам на компьютер и самое главное - как от него защититься?

\section*{ОСНОВНАЯ ЧАСТЬ}
\addcontentsline{toc}{section}{ОСНОВНАЯ ЧАСТЬ}
\subsection*{Что такое компьютерный вирус}
\addcontentsline{toc}{subsection}{Что такое компьютерный вирус}

Согласно определению, \textbf{компьютерный вирус} - вид вредоносного программного обеспечения, способного внедряться в код других программ, системные области памяти, загрузочные секторы, и распространять свои копии по каналам связи.

\textbf{Основная задача} вируса - его \textbf{\textit{распространение}}.

Помимо кражи данных, компьютерные вирусы также могут нарушать работу компьютера путём:
\begin{itemize}
    \item изменения названия или подмена расширения файлов (например, все файлы \textit{.exe} меняются на \textit{.jpg})
    \item заполнения оперативной и встроенной памяти
    \item повреждения системно-важных частей
    \item аппаратного нарушение работы компьютера (увеличение температуры ЦП и ГП, снижение скорости вращения кулеров и т.д.)
    \item шифрования всей информации на компьютере
\end{itemize}

\subsection*{Классификация компьютерных вирусов}
\addcontentsline{toc}{subsection}{Классификация компьютерных вирусов}

В настоящее время специалисты по компьютерной вирусологии ещё не выработали единую классификацию для вирусов. Но всё же можно выделить несколько пунктов по которым мы сможем систематизировать их [4].

Компьютерные вирусы различаются по:

\begin{itemize}
    \item Операционной системе
    \item Среде обитания
    \item Способу заражения
    \item По <<опасности>> вируса
    \item Способу заражения файлов
\end{itemize}

Далее разберём каждый из вышеуказанных пунктов отдельно.

\subsubsection*{Операционная система}
\addcontentsline{toc}{subsubsection}{Операционная система}
На данный момент существуют несколько операционных систем: Windows, macOS и Linux - для компьютеров, Android и iOS - для мобильных устройств. И поскольку архитектуры у этих систем совершенно разные, то и вирусы для этих ОС разные и работают по-разному.

\subsubsection*{Среда обитания}
\addcontentsline{toc}{subsubsection}{Среда обитания}
Под <<средой обитания>> понимается то, куда внедряется вирус. По среде обитания разделяют на:

\begin{itemize}
    \item Файловые вирусы 
        \begin{flushleft}
        Попадая на устройство, внедряется в различные файлы операционной системы этого устройства [5].
        \end{flushleft}
    \item Загрузочные вирусы
        \begin{flushleft}
        При попадании на устройство выделяют место на компьютере, недоступное для поиска, и при каждом запуске ОС, запускают себя [6].
        \end{flushleft}
    \item Макро-вирусы
        \begin{flushleft}
        Встроены в графические и текстовые системы обработки (например, MS Word) [7].
        \end{flushleft}
    \item Скрипт-вирусы
        \begin{flushleft}
        Находясь в уязвимом месте программы, заставляют её совершать несвойственные ей действия [8].
        \end{flushleft}
\end{itemize}

\subsubsection*{Способ заражения}
\addcontentsline{toc}{subsubsection}{Способ заражения}
Под способом заражения понимается то, каким образом вирус заражает устройства. Различают:

\begin{itemize}
    \item Резидентные вирусы
    \begin{flushleft}
        При попадании на устройство оставляют свою резидентную часть в памяти и остаются активным от включения до выключения устройства.
    \end{flushleft}
    \item Нерезидентные
    \begin{flushleft}
        Данный тип оставляет на устройстве нерезидентную часть, не способную к размножению.
    \end{flushleft}
\end{itemize}

\subsubsection*{<<Опасность вируса>>}
\addcontentsline{toc}{subsubsection}{<<Опасность вируса>>}
Под <<опасностью>> имеется ввиду, какой вред может причинить вирус устройству. По данному пункту различают:

\begin{itemize}
    \item Неопасные вирусы
        \begin{flushleft}
        Данные вирусы зачастую направлены на заполнение памяти компьютера, поэтому их ущерб минимален.
        \end{flushleft}
    \item Безвредные вирусы
        \begin{flushleft}
        Данные вирусы заполняют как встроенную, так и оперативную память, заставляя устройство работать чуть медленнее обычного, но никакого сильного влияния на работу устройства не оказывают.
        \end{flushleft}
    \item Вредные
        \begin{flushleft}
            Данные вирусы уже могут причинять более существенный вред компьютеру помимо потребления ресурсов устройства.
        \end{flushleft}
    \item Опасные
        \begin{flushleft}
            Данные вирусы являются самыми опасными, потому что их воздействие на устройство может привести к потере данных и программ.
        \end{flushleft}
\end{itemize}

\subsubsection*{Способ заражения файлов}
\addcontentsline{toc}{subsubsection}{Способ заражения файлов}
Под способом заражения имеется ввиду, каким образом вирус внедряется в файлы на устройстве.

\textbf{Перезаписывающие} \l
Вирус полностью перезаписывает свой код вместо кода заражённого файла. Естественно, что файлы не подлежат восстановлению. Обнаруживаются из-за нарушения работы операционной и файловой систем из-за отсутствия необходимых компонентов, которые были перезаписаны.

\textbf{Паразитические} \l
Вирусы, распространяясь на устройстве, внедряются в файлы путём изменения содержимого файлов. Однако работоспособность файлов остаётся такой же или немного ухудшается. Основные типы таких вирусов:

\begin{itemize}
    \item внедряющиеся в начало
        \begin{flushleft}
            Вирус помещается в начало файла, тем самым запускаясь первее, чем код основной программы.
        \end{flushleft}
    \item внедряющиеся в середину
        \begin{flushleft}
            Аналогично помещению в начало или конец кода. Некоторые вирусы при этом могут компрессировать переносимый блок файла без изменения длины.
        \end{flushleft}
    \item внедряющиеся в конец
        \begin{flushleft}
            Вирус помещается в конец файла, тем запускаясь в последнюю очередь.
        \end{flushleft}
\end{itemize}

\subsection*{Разновидность вирусов}
\addcontentsline{toc}{subsection}{Разновидость вирусов}
Вирусы различаются по своим целям и способам проникновения в компьютеры жертв. Одни заполняют память компьютера, заставляя его работать медленнее, а другие позволяют иным вирусам проникнуть на устройство незамеченными. Разберём некоторые из них:

\subsubsection*{Черви}
\addcontentsline{toc}{subsubsection}{Черви}
\textbf{Цель} данного вируса - заполнение памяти компьютера различным <<мусором>> с целью замедлить его работу. Может размножаться, но при этом не является частью како-либо программы. [12].

\subsubsection*{Троян}
\addcontentsline{toc}{subsubsection}{Троян}
\textbf{Цель} - проникновение на устройство для дальнейшей кражи данных. Попадает на устройство в виде программного обеспечения или его компонента. Начинает свою работу только при запуске самого программного обеспечения, поэтому до запуска нанести вред \textbf{\textit{не может}}. [13]

\subsubsection*{Бэкдор (программа-шпион)}
\addcontentsline{toc}{subsubsection}{Бэкдор (программа-шпион)}
Бэкдор \textit{(от англ. backdoor - чёрный вход)} - алгоритм, дающий злоумышленникам удалённо управлять устройством жертвы. \textbf{Цель} - кража конфиденциальной информации, а также дальнейшее проникновение. Например, дроппер на компьютере работника какой-либо фирмы позволяет получить доступ к системе этой фирмы. [9]

\subsubsection*{Эксплойт}
\addcontentsline{toc}{subsubsection}{Эксплойт}
\textbf{Эксплойт} \textit{(от англ. exploit - эксплуатировать} - программы, использующие уязвимости программ для проникновения на устройство жертвы. \textbf{Цель} - получение доступа к устройству с последующей кражей конфиденциальных данных. [10]

\subsubsection*{Дроппер}
\addcontentsline{toc}{subsubsection}{Дроппер}
\textbf{Дроппер} \textit{(от англ. dropper - <<бомбосбрасыватель>>)} - программа (семейства троянов), предназначенная для скрытой установки других, более опасных вирусных программ. \textbf{Цель} - установка других вредоносных программ без возможности обнаружения. [11]


\subsection*{Обнаружение вирусов и методы защиты против них}
\addcontentsline{toc}{subsection}{Обнаружение вирусов и методы защиты против них}

Как понять, что ваш компьютер заражен вредоносной программой? 
Обнаружить это можно по следующим признакам:

\begin{itemize}
    \item неправильное поведение ранее корректно работающих программ
    \item сильное замедление работы компьютера
    \item операционная система загружается с ошибками или не загружается вообще
    \item искажение файлов, изменение их размеров или увеличение их количества в памяти
    \item уменьшение количества свободной оперативной памяти
    \item сбои в работе компьютера
    \item аномальное повышение температуры компонентов компьютера
\end{itemize}

Итак, вы заметили, что несколько признаков из данного списка присутствуют на вашем компьютере. Теперь нужно понять как избавиться от <<зловреда>>.

Ни в коем случае не нужно самостоятельно пытаться найти и удалить данную программу - это не поможет. В настоящее время от вирусов нельзя избавиться простым удалением. Нужна более глубокая чистка компьютера - от файловой системы до реестра. С этим могут справиться только \textbf{антивирусные программы}.

Они так же, как и вирусы, имеют несколько разновидностей [14]. Каждая предназначена для определённой задачи.
Существуют:

\begin{itemize}
    \item программы-детекторы
    \item программы-доктора
    \item программы-ревизоры
    \item программы-фильтры
    \item программы-вакцины
\end{itemize}

Разберём каждую из них отдельно.

\subsubsection*{Программы-детекторы}
\addcontentsline{toc}{subsubsection}{Программы-детекторы}
Осуществляют поиск вируса по его \textbf{сигнатуре}.

Согласно определению, \textbf{сигнатура вируса} - характерные признаки вируса, используемые для его обнаружения. Этими признаками может быть как строки кода, последовательность байтов вируса, так и его поведение на устройстве [3].

\subsubsection*{Программы-доктора}
\addcontentsline{toc}{subsubsection}{Программы-доктора}

Находят и лечат заражённые файлы, возвращая их в первоначальное состояние. Сначала ищут вредоносное ПО в оперативной памяти, затем ищут в файлах, хранящихся в памяти устройства

\subsubsection*{Программы-ревизоры}
\addcontentsline{toc}{subsubsection}{Программы-ревизоры}

Они запоминают первоначально-зафиксированные параметры файлов на устройстве и периодически сверяют текущее состояние с исходным с целью обнаружить изменения, которые внёс вирус. Параметры, фиксирующиеся программой-детектором:

\begin{itemize}
    \item длина
    \item вес
    \item \textit{контрольная сумма}
    \item дата, время модификаций
    \item др. параметры
\end{itemize}

Согласно определению, \textbf{контрольная сумма} - некоторое значение, рассчитанное по набору данных путём применения определённого алгоритма и используемое для проверки целостности данных при их передаче или хранении [2].

\subsubsection*{Программы-фильтры}
\addcontentsline{toc}{subsubsection}{Программы-фильтры}
Следят за действиями, которые происходят на устройстве, и фиксируют подозрительные операции. Если совершённая операция показалась программе странной и пользователь не знал о ней, то фильтр блокирует выполнение операции, а пользователь в свою очередь узнаёт о присутствии вируса.

\subsubsection*{Программы-вакцины}
\addcontentsline{toc}{subsubsection}{Программы-вакцины}
Полностью предотвращают распространение вируса и применяются в тех случаях, когда нет программ-докторов, способных избавится от вируса.

\subsection*{Защита от вирусов}
\addcontentsline{toc}{subsection}{Защита от вирусов}
Чтобы устройство снова не подверглось \textbf{вирусной атаке}, нужно соблюдать некоторые правила:

\begin{enumerate}
    \item Установить на своё устройство антивирусную программу
    \item Перед считыванием информации с флешки или диска всегда проверять их на наличие вирусов
    \item При разархивации различных файлов, проверять архивы на наличие вирусов
    \item Периодически проверять жёсткие диски на наличие зловреда
    \item Проверяйте все скачиваемые с Интернета файлы, так как именно оттуда чаще всего можно получить вредоносную программу
    \item Регулярно делать резервную копию важных данных
\end{enumerate}

Соблюдение данных правил не даёт гарантию защиты от вредоносных программ, но значительно снижает шанс их проникновение на устройство.

\newpage
\section*{ЗАКЛЮЧЕНИЕ}
\addcontentsline{toc}{section}{ЗАКЛЮЧЕНИЕ}

История приводит нам много случаев угрозы информационным ресурсам. Число вирусов, как и их опасность, растёт. Алгоритмы программ становятся сложнее, их труднее обнаружить и избавиться от них. И в связи с этим разработчикам антивирусных программ приходится подстраиваться под различные ситуации и совершенствовать антивирусные программы.

\addcontentsline{toc}{subsection}{Список литературы}
\begin{thebibliography}{11}

\bibitem{WIKI1}
Wikipedia // Компьютерный вирус // \href{https://en.wikipedia.org/wiki/Computer_virus}{\underline{Ссылка на источник}}

\bibitem{WIKI2}
Wikipedia // Контрольная сумма // \href{https://en.wikipedia.org/wiki/Checksum}{\underline{Ссылка на источник}}

\bibitem{TCI}
TCI (Технический центр Интернет) // Сигнатура вируса // \href{https://www.tcinet.ru/press-centre/glossary/article.php?ELEMENT_ID=5129}{\underline{Ссылка на источник}}

\bibitem{TADVISER}
TADVISER // Классификация вирусов // \href{https://goo-gl.ru/SMmIV}{\underline{Ссылка на источник}}

\bibitem{WIKI3}
ДиалогНаука // Файловый вирус // \href{https://www.dialognauka.ru/support/golossary/4572/}{\underline{Ссылка на источник}}

\bibitem{WIKI4}
ДиалогНаука // Загрузочный вирус // \href{https://www.dialognauka.ru/support/golossary/4574/?sphrase_id=61810}{\underline{Ссылка на источник}}

\bibitem{WIKI5}
Wikipedia // Макровирус // \href{https://goo-gl.ru/bncO9}{\underline{Ссылка на источник}}

\bibitem{VUZLIT}
VUZLIT // Скрипт-вирус // \href{https://vuzlit.ru/1022038/skript_virusy}{\underline{Ссылка на источник}}

\bibitem{WIKI6}
Wikipedia // Бэкдор // \href{https://goo-gl.ru/FAcyf}{\underline{Ссылка на источник}}

\bibitem{WIKI7}
Wikipedia // Эксплойт // \href{https://goo-gl.ru/f9X1d}{\underline{Ссылка на источник}}

\bibitem{WIKI8}
Wikipedia // Дроппер // \href{https://goo-gl.ru/NCbCc}{\underline{Ссылка на источник}}

\bibitem{WIKI9}
Wikipedia // Червь // \href{https://goo-gl.ru/6PGEf}{\underline{Ссылка на источник}}

\bibitem{WIKI10}
Wikipedia // Троян // \href{https://goo-gl.ru/dPriv}{\underline{Ссылка на источник}}

\bibitem{WIKI11}
Wikipedia // Антивирусы // \href{https://goo-gl.ru/5XWi}{\underline{Ссылка на источник}}
\end{thebibliography}

\end{document}