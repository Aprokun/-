\documentclass{article}
\usepackage[russian]{babel}
\usepackage[utf8]{inputenc}
\usepackage{setspace}
\usepackage{titletoc}

\linespread{1,5}
\def\l{\linebreak}
\def\contentsname{СОДЕРЖАНИЕ}

\begin{document}
\thispagestyle{empty}

\begin{center}
    \textbf{МИНИСТЕРСТВО ОБРАЗОВАНИЯ И НАУКИ
    РОССИЙСКОЙ ФЕДЕРАЦИИ} \l
    ФЕДЕРАЛЬНОЕ ГОСУДАРСТВЕННОЕ БЮДЖЕТНОЕ ОБЩЕОБРАЗОВАТЕЛЬНОЕ УЧРЕЖДЕНИЕ
    ВЫСШЕГО ОБРАЗОВАНИЯ \l
    \textbf{<<БЕЛГОРОДСКИЙ ГОСУДАРСТВЕННЫЙ
    ТЕХНОЛОГИЧЕСКИЙ УНИВЕРСИТЕТ им В.Г.ШУХОВА>> \l
    (БГТУ им. В.Г.Шухова)}
    \l \l \l
    Кафедра программного обеспечения вычислительной техники и 
    автоматизированных систем
    \l \l \l \l
    Расчётно-графическое задание \l
    дисциплина: Информатика \l
    тема: <<Видеокарты>> \l
\end{center}
\hfill
\begin{flushright}
    Выполнил: ст.группы ПВ-201 \l
    Машуров Дмитрий Русланович \l
    Проверил: Бондаренко Т.В. \l
\end{flushright}
\vfill
\begin{center}
    Белгород 2020
\end{center}


\newpage
\thispagestyle{empty}


\newpage
\setcounter{page}{1}

\begin{center}
    \section*{\Large\textbf{ВВЕДЕНИЕ}}
\end{center}

\subsection*{Постановка темы}
Lorem ipsum dolor sit amet, consectetur adipiscing elit, sed do eiusmod tempor incididunt ut labore et dolore magna aliqua. Ut enim ad minim veniam, quis nostrud exercitation ullamco laboris nisi ut aliquip ex ea commodo consequat. Duis aute irure dolor in reprehenderit in voluptate velit esse cillum dolore eu fugiat nulla pariatur. Excepteur sint occaecat cupidatat non proident, sunt in culpa qui officia deserunt mollit anim id est laborum

\subsection*{Актуальность темы}
Lorem ipsum dolor sit amet, consectetur adipiscing elit, sed do eiusmod tempor incididunt ut labore et dolore magna aliqua. Ut enim ad minim veniam, quis nostrud exercitation ullamco laboris nisi ut aliquip ex ea commodo consequat. Duis aute irure dolor in reprehenderit in voluptate velit esse cillum dolore eu fugiat nulla pariatur. Excepteur sint occaecat cupidatat non proident, sunt in culpa qui officia deserunt mollit anim id est laborum

\newpage
\begin{center}
  \section*{\Large\textbf{Основная часть}}
\end{center}
\subsection*{\largeЭто название главы 1}
текст \textit{главы 1}
\subsection*{\largeЭто название главы 2}
текст \textit{главы 2}

\newpage
\begin{center}
    \section*{\Large\textbf{ЗАКЛЮЧЕНИЕ}}
\end{center}
\subsection*{Глава заключения}
тут я подвожу итоги по проделанной работе

\newpage

\begin{center}
    \subsection*{\textbf{Список литературы:}}
\end{center}

\begin{enumerate}
\item\textbf{Для книги:}

Баженов Ю.М. Технология бетона. М.: Изд-во АСВ, 2002. 500 с.

\item\textbf{Для статей в журналах:}

\textbf{\textit{До 3 авторов}}

Клюев С.В., Лесовик Р.В. Дисперсно-армированный мелкозернистый бетон с использованием полипропиленового волокна // Бетон и железобетон. 2011. №3. С. 7–9.

\textbf{\textit{Более 3 авторов (авторы перечисляются в полном составе)}} \\
Лесовик В.С., Алфимова Н.И., Яковлев Е.А., Шейченко М.С. К проблеме повышения эффективности композиционных вяжущих // Вестник Белгородского государственного технологического университета им. В.Г. Шухова. 2009. №1. С. 30–33.

\item\textbf{Для электронной публикации:}

Булатов Г.Я. Проектирование технологии общестроительных работ [Электронный ресурс]. URL: ftp://ftp.unilib.neva.ru/dl/137.pdf.

\item\textbf{Ссылки на статьи в сборниках трудов:}

\textbf{\textit{До 3 авторов}}

Алфимова Н.И., Черкасов В.С. К проблеме оценки пригодности техногенного сырья для производства строительных материалов / Наука и молодежь в начале нового столетия: сб. материалов конф. III Междунар. науч.-практ. конф. студентов, аспирантов и молодых ученых // Губкинский филиал Белгор. гос. технол. ун-та. (Губкин, 8–9 апр. 2010 г.), Губкин: Изд-во БГТУ, 2010. С. 31-33.

\textbf{\textit{Более 3 авторов}}

Алфимова Н.И., Вишневская Я.Ю., Черкасов В.С., Шаповалов Н.Н. Повышение эффективности композиционных вяжущих за счет использования отходов производства керамзита и оптимизации режимов твердения // Научные исследования, наносистемы и ресурсосберегающие технологии в промышленности строительных материалов (XIX Научные чтения): Междунар. науч.-практ. конф., (Белгород, 5–8 окт. 2010 г.), Белгород: Изд-во БГТУ, 2010. Ч.1. С. 36-38.

\item\textbf{Патенты:}

Пат. 2329361 Российская Федерация, МПК7 Е 04 С 3/08. Узловое бесфасоночное соединение трубчатых элементов фермы
\end{enumerate}

\end{document}













